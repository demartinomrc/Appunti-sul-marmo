% !TEX encoding = UTF-8
% !TEX TS-program = pdflatex
% !TEX root = ../Tesi.tex
% !TEX spellcheck = it-IT

%*******************************************************
% Dedica
%*******************************************************
%\cleardoublepage
\phantomsection
\thispagestyle{empty}
\pdfbookmark{Dedica}{Dedica}

~

\vfill

\begin{flushright}
{\footnotesize
C’è gente che trova figure \\
nascoste nella carta da parati \\
o nelle nuvole. \\
A me succede lo stesso coi rumori. \\
Per essere più esatti, ho un vecchio phon \\
che appena si accende comincia a vibrare \\
e man mano \\
emette un lamento profondo. \\
E’ l’elica difettosa, o i cuscinetti a sfera, \\
non ne ho idea, \\
ma so che inizia a intonare una trenodia, \\
o meglio, a sussurrarla sottovoce. \\
Prima si avvertono solo suoni indistinti, \\
una folla che fugge, moto che si avvicinano, \\
ma facendo attenzione \\
appaiono via via urla, richiami. \\
Io mi concentro; una sera, addirittura, \\
sono arrivato a bruciarmi, tale è lo sforzo \\
per afferrare il groviglio, il nodo acustico \\
dell’asciugacapelli. \\
Perché il suo sferragliare non resta sempre uguale: \\
più dura, più si sciolgono gli intrecci \\
del fragore, le voci si distinguono. \\
Sento dialetti slavi, minacce, spesso spari: \\
un giorno sono rimasto ad ascoltarlo quasi dieci minuti \\
per seguire la fasi di un rastrellamento \\
in un lontano villaggio dei Balcani. \\
A volte ne esce uno squillo familiare, \\
credo che sia il telefono, spengo, \\
vado a rispondere, \\
ma non c’é mai nessuno: quei segnali, \\
si vede che provengono da un’altra parte, \\
sempre. \\
Se qualcuno ti chiama, non ci credere,\\
sarà un miraggio uditivo, un’impressione. \\
La verità è diversa: \\
mentre mi punto alla tempia quell’attrezzo \\
che sembra una pistola, \\
viene fuori il racconto di storie terribili, \\
fucilazioni, il pianto di bambini. \\
E’ come una confessione non richiesta, \\
una registrazione spedita per errore. \\
Che c’entro, io, con tutto questo sangue, \\
io che mi voglio solo asciugare la testa? \\
Ormai ci penso due volte, prima di adoperarlo, \\
prima di sprofondare in quell’orrore \\
e assistere impotente a certe scene. \\
Meglio bagnato, allora. \\
Mi verrà il torcicollo? Poco male} \\ \medskip
--- Valerio Magrelli%\footnote{Valerio Magrelli (Roma, 1957),  da Il sangue amaro (Einaudi, 2014)}    

\end{flushright}